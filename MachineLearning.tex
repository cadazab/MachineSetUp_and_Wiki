\documentclass[english]{article}
\usepackage[latin9]{inputenc}
\usepackage{geometry}
\geometry{verbose,tmargin=2.6cm,bmargin=2.6cm,lmargin=2.6cm,rmargin=2.6cm,
headheight=1.3cm,headsep=1.3cm,footskip=1.3cm}
\usepackage{amsmath}
\usepackage{graphicx}
\usepackage{nomencl}
\usepackage{amsfonts}
% the following is useful when we have the old nomencl.sty package
\providecommand{\printnomenclature}{\printglossary}
\providecommand{\makenomenclature}{\makeglossary}
\makenomenclature
\usepackage{babel}
\begin{document}

\part*{Introduction}


\section{What is Machine Learning?}
Two definitions of Machine Learning are offered. Arthur Samuel described it as:
"the field of study that gives computers the ability to learn without being
explicitly programmed." This is an older, informal definition.\\

Tom Mitchell provides a more modern definition: "A computer program is said to
learn from experience E with respect to some class of tasks T and performance
measure P, if its performance at tasks in T, as measured by P, improves with
experience E."\\

\textbf{Example:} playing checkers.

\begin{itemize}
\item $E =$ the experience of playing many games of checkers
\item $T =$ the task of playing checkers.
\item $P =$ the probability that the program will win the next game.
\end{itemize}

\section{Supervised Learning}
In supervised learning, we are given a data set and already know what our
correct output should look like, having the idea that there is a relationship
between the input and the output.\\

Supervised learning problems are categorized into "\textbf{regression}" and
"\textbf{classification}" problems. In a regression problem, we are trying to
predict results within a \textbf{continuous} output, meaning that we are trying
to map input variables to some \textbf{continuous} function. In a
\textbf{classification} problem, we are instead trying to predict results in a
\textbf{discrete} output. In other words, we are trying to map input variables
into \textbf{discrete} categories.\\

\textbf{Example:}\\

Given data about the size of houses on the real estate market, try to predict
their price. Price as a function of size is a \textit{continuous} output, so
this is a regression problem.

We could turn this example into a classification problem by instead making our
output about whether the house "sells for more or less than the asking price."
Here we are classifying the houses based on price into two \textit{discrete}
categories.

\section{Unsupervised Learning}
Unsupervised learning, on the other hand, allows us to approach problems with
little or no idea what our results should look like. We can derive structure
from data where we don't necessarily know the effect of the variables.\\

We can derive this structure by clustering the data based on relationships among
the variables in the data.\\

With unsupervised learning there is no feedback based on the prediction results,
i.e., there is no teacher to correct you. It is not just about clustering.
For example, associative memory is unsupervised learning.\\

\textbf{Example:}\\

\textit{Clustering:} Take a collection of 1000 essays written on the US Economy,
and find a way to automatically group these essays into a small number that are
somehow similar or related by different variables, such as word frequency,
sentence length, page count, and so on.

\textit{Associative:}  Suppose a doctor over years of experience forms
associations in his mind between patient characteristics and illnesses that
they have. If a new patient shows up then based on this patients characteristics
such as symptoms, family medical history, physical attributes, mental outlook,
etc the doctor associates possible illness or illnesses based on what the doctor
has seen before with similar patients. This is not the same as rule based
reasoning as in expert systems. In this case we would like to estimate a mapping
function from patient characteristics into illnesses.

\section{Training Set}

For $Y^{1}$, a dependent variable of the independent variables
$x^{1}_{1},x^{1}_{2},...,x^{1}_{n}$, which shape the vector $X^{1}$, we define
the Training Set $T$ as the group of elements
$T=(X^{1},Y^{1},X^{2},Y^{2},...,X^{m},Y^{m})$, where $m,n \in \mathbb{N}$. $m$
is the number of \textbf{Training examples}, and $n$ is the number of
independent variables associated with each dependent variable, or the number of
 \textbf{Features}. hereinafter we will call the $X^{j}$ like \textbf{Inputs},
 and the $Y^{j}$ like \textbf{Outputs}.\\

\textbf{Example:}
Suppose the price of a house depends on several features like it is shown in the
following table:

\begin{center}
\begin{tabular}{ |c|c|c|c|c| }
\hline
Size($feet^{2}$) & Number of bedrooms & Number of floors & Age($years$) &
Price$(\$1000)$ \\
\hline
2104 & 5 & 1 & 45 & 460 \\
1416 & 3 & 2 & 40 & 232 \\
1534 & 3 & 2 & 30 & 315 \\
852  & 2 & 1 & 36 & 178 \\
\hline
\end{tabular}
\end{center}

For this example we have $m=4$(Features) and $n=4$(Training examples). And

$X^{1}=
\begin{bmatrix}
2104, & 5, & 1, & 45
\end{bmatrix},
Y^{1}=
\begin{bmatrix}
460
\end{bmatrix},
$ and $X^{2}_{3}=2$

\section{Lineal Regression}
For this regression we suppose that it is possible to get an output value
$h_{\theta}(x)$, fitting the input values in a lineal function that is call
\textbf{Hypothesis}:

\begin{equation}
h_{\theta}(x)=\theta_{0}x_{0}+\theta_{1}x_{2}+\theta_{2}x_{2}+...+\theta_{n}x_{n}
\end{equation}

Where $\theta_{0},\theta_{1},\theta_{2},...,\theta_{n}$ are the
\textbf{Parameters} of the \textit{hypothesis}.

\subsection{Cost function}
If we will find some function to fit the input parameters and obtain some output,
it is useful to compare the results with the training $Y$ values, to do that we
can use the cost function:

\begin{equation}
J(\theta_{0},\theta_{1},\theta_{2},...,\theta_{n})=
\frac{1}{2m}\sum_{i=1}^{m}(h_{\theta}(x^{(i)})-y^{(i)})^{2}
\end{equation}

The matrix version of this formula is:

\begin{equation}
J(\theta)=\frac{1}{2m}(X\theta-Y)^{T}(X\theta-Y)
\end{equation}

The purpose is obtain a function so that the value of the cost function is the
lowest possible, to find these value there are different techniques, one of them
is the gradient descent.

\subsection{Gradient Descent}
The idea is to start at some initial value for the hypothesis function, find the
direction in which the function decreases faster(partial derivative), and take
a small step in that direction(multiply by a small constant), and then repeat
the process until find the local minimum value of the function. One step is:

\begin{equation}
\theta_{j}:=\theta_{j}-\alpha \frac{\partial}{\partial \theta_{j}}J=
\theta_{j}-\alpha \frac{1}{m}
\sum_{i=1}^{m}(h_{\theta}(x^{(i)})-y^{(i)}) x_{j}^{(i)}
\end{equation}

\subsection{Feature Normalization}
When features differ by orders of magnitude, first performing feature scaling
can make gradient descent converge much more quickly. The most general formula
to normalize the feature values is:

\begin{equation}
x_{j}^{i} =: \frac{ x_{j}^{i}-\mu}{s}
\end{equation}

Where $\mu$ is the average value, and $\mu$ is the standard
deviation of $X^{(i)}$.

\subsection{Normal Equation}
There is a closed-from solution of to linear regression:

\begin{equation}
  \theta = (X^{T}X)^{-1}X^{T}Y
\end{equation}

\end{document}
